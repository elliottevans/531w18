\documentclass[]{article}
\usepackage{lmodern}
\usepackage{amssymb,amsmath}
\usepackage{ifxetex,ifluatex}
\usepackage{fixltx2e} % provides \textsubscript
\ifnum 0\ifxetex 1\fi\ifluatex 1\fi=0 % if pdftex
  \usepackage[T1]{fontenc}
  \usepackage[utf8]{inputenc}
  \usepackage{eurosym}
\else % if luatex or xelatex
  \ifxetex
    \usepackage{mathspec}
  \else
    \usepackage{fontspec}
  \fi
  \defaultfontfeatures{Ligatures=TeX,Scale=MatchLowercase}
  \newcommand{\euro}{€}
\fi
% use upquote if available, for straight quotes in verbatim environments
\IfFileExists{upquote.sty}{\usepackage{upquote}}{}
% use microtype if available
\IfFileExists{microtype.sty}{%
\usepackage{microtype}
\UseMicrotypeSet[protrusion]{basicmath} % disable protrusion for tt fonts
}{}
\usepackage[margin=1in]{geometry}
\usepackage{hyperref}
\hypersetup{unicode=true,
            pdftitle={9 Introduction to partially observed Markov process models},
            pdfauthor={Edward Ionides},
            pdfborder={0 0 0},
            breaklinks=true}
\urlstyle{same}  % don't use monospace font for urls
\usepackage{graphicx,grffile}
\makeatletter
\def\maxwidth{\ifdim\Gin@nat@width>\linewidth\linewidth\else\Gin@nat@width\fi}
\def\maxheight{\ifdim\Gin@nat@height>\textheight\textheight\else\Gin@nat@height\fi}
\makeatother
% Scale images if necessary, so that they will not overflow the page
% margins by default, and it is still possible to overwrite the defaults
% using explicit options in \includegraphics[width, height, ...]{}
\setkeys{Gin}{width=\maxwidth,height=\maxheight,keepaspectratio}
\IfFileExists{parskip.sty}{%
\usepackage{parskip}
}{% else
\setlength{\parindent}{0pt}
\setlength{\parskip}{6pt plus 2pt minus 1pt}
}
\setlength{\emergencystretch}{3em}  % prevent overfull lines
\providecommand{\tightlist}{%
  \setlength{\itemsep}{0pt}\setlength{\parskip}{0pt}}
\setcounter{secnumdepth}{0}
% Redefines (sub)paragraphs to behave more like sections
\ifx\paragraph\undefined\else
\let\oldparagraph\paragraph
\renewcommand{\paragraph}[1]{\oldparagraph{#1}\mbox{}}
\fi
\ifx\subparagraph\undefined\else
\let\oldsubparagraph\subparagraph
\renewcommand{\subparagraph}[1]{\oldsubparagraph{#1}\mbox{}}
\fi

%%% Use protect on footnotes to avoid problems with footnotes in titles
\let\rmarkdownfootnote\footnote%
\def\footnote{\protect\rmarkdownfootnote}

%%% Change title format to be more compact
\usepackage{titling}

% Create subtitle command for use in maketitle
\newcommand{\subtitle}[1]{
  \posttitle{
    \begin{center}\large#1\end{center}
    }
}

\setlength{\droptitle}{-2em}
  \title{9 Introduction to partially observed Markov process models}
  \pretitle{\vspace{\droptitle}\centering\huge}
  \posttitle{\par}
  \author{Edward Ionides}
  \preauthor{\centering\large\emph}
  \postauthor{\par}
  \predate{\centering\large\emph}
  \postdate{\par}
  \date{2/27/2018}


\begin{document}
\maketitle

{
\setcounter{tocdepth}{2}
\tableofcontents
}
\newcommand\prob{\mathbb{P}}
\newcommand\E{\mathbb{E}}
\newcommand\var{\mathrm{Var}}
\newcommand\cov{\mathrm{Cov}}
\newcommand\loglik{\ell}
\newcommand\R{\mathbb{R}}
\newcommand\data[1]{#1^*}
\newcommand\params{\, ; \,}
\newcommand\transpose{\scriptsize{T}}
\newcommand\eqspace{\quad\quad}
\newcommand\myeq[1]{\eqspace \displaystyle #1}
\newcommand\lik{\mathscr{L}}
\newcommand\profileloglik[1]{\ell^\mathrm{profile}_#1}
\newcommand\ar{\phi}
\newcommand\ma{\psi}
\newcommand\AR{\Phi}
\newcommand\MA{\Psi}
\newcommand\ev{u}
\newcommand\given{{\, | \,}}
\newcommand\equals{{=\,}}
\newcommand\matA{\mathbb{A}}
\newcommand\matB{\mathbb{B}}
\newcommand\matH{\mathbb{H}}
\newcommand\covmatX{\mathbb{U}}
\newcommand\covmatY{\mathbb{V}}





\begin{center}\rule{0.5\linewidth}{\linethickness}\end{center}

\begin{center}\rule{0.5\linewidth}{\linethickness}\end{center}

Objectives

\begin{itemize}
\item
  Develop a framework for thinking about models that consist of a
  stochastic dynamic system observed with noise.
\item
  In the linear Gaussian case, develop matrix operations to find an
  exact and computationally fast algorithm for the likelihood function.
  This algorithm is called the \textbf{Kalman filter}.
\item
  Understand how the Kalman filter is used to compute the likelihood for
  ARMA models.
\item
  See how the Kalman filter also facilitates forecasting and estimation
  of the state of the unobserved process.
\item
  Start to investigate the general nonlinear filtering equations.
\end{itemize}

\begin{center}\rule{0.5\linewidth}{\linethickness}\end{center}

\begin{center}\rule{0.5\linewidth}{\linethickness}\end{center}

\subsection{Partially observed Markov processes (POMP)
models}\label{partially-observed-markov-processes-pomp-models}

\begin{itemize}
\item
  Uncertainty and variability are ubiquitous features of processes in
  the biological and social sciences. A physical system obeying Newton's
  laws is fully predictable, but complex systems are in practice not
  perfectly predictable---we can only forecast weather reliably in the
  near future.
\item
  Basic time series model of deterministic trend plus colored noise
  imply perfect reproducibility and (so far as the deterministic trend
  model can be extrapolated) forecasts that remain acurate far into the
  future.
\item
  To model variability and unpredictability in low frequency components,
  we may wish to specify a random process model for how the system
  evolves. We could call this a ``stochastic trend'' approach, though
  that is an oxymoron since we've defined trend to be expected value.
\item
  As in the deterministic signal plus noise model, we will model the
  observations as random variables conditional on the trajectory of the
  latent process.
\item
  The model for how the latent Markov process evolves is therefore
  called a \textbf{latent process model} or a \textbf{hidden process
  model} to acknowledge that we are modeling a process that we suppose
  cannot directly be observed. It is also sometimes called a
  \textbf{state process} since we may think of the latent process as
  representing the indirectly measured state of some system.
\item
  A standard class of latent process models is characterized by the
  requirement that the future evolution of the system depends only on
  the current state, plus randomness introduced in future. A model of
  this type is called a \textbf{Markov chain}, or, if the process is
  defined in continuous time rather than just at discrete time points, a
  \textbf{Markov process}. We will use the term Markov process for both
  discrete and continous time.
\item
  \textbf{Partial observations} here mean either or both of (i)
  measurement noise; (ii) entirely unmeasured latent variables. Both
  these features are present in many systems.
\item
  To specify a \textbf{partially observed Markov process} (POMP) model,
  we will need to specify a latent Markov process model. We must also
  then specify how the \textbf{observation process model} is supposed to
  generate the data given the latent process.
\item
  In a POMP model, the latent Markov process model can also be called
  the \textbf{state process model} or the \textbf{hidden process model}.
\item
  Often, much of the scientific interest is in understanding what models
  for the behavior of this latent process are consistent with the data.
\item
  A good model for the underlying, but imperfectly observed, dynamics of
  a system can also lead to a skillful forecast.
\item
  We are going to introduce a general framework for specifying POMP
  models. This generality will give us the flexibility to develop models
  and methods appropriate to a range of applications.
\end{itemize}

\begin{center}\rule{0.5\linewidth}{\linethickness}\end{center}

\begin{center}\rule{0.5\linewidth}{\linethickness}\end{center}

\subsubsection{Discrete time Markov
processes}\label{discrete-time-markov-processes}

\begin{itemize}
\tightlist
\item
  A time series model \(X_{0:N}\) is a \textbf{Markov process} model if
  the conditional densities satisfy the \textbf{Markov property} that,
  for all \(n\in 1:N\).
\end{itemize}

{[}MP1{]}
\(\eqspace f_{X_n|X_{1:n-1}}(x_n\given x_{1:n-1}) = f_{X_n|X_{n-1}}(x_n\given x_{n-1})\),

\begin{itemize}
\item
  We suppose that the random process \(X_n\) occurs at time \(t_n\) for
  \(n\in 0:N\), so the discrete time process corresponds to time points
  in continuous time.
\item
  We have \textbf{initialized} the Markov process model at a time
  \(t_0\), although we will suppose that data are collected only at
  times \(t_{1:N}\).

  \begin{itemize}
  \item
    The initialization model could be deterministic (a fixed value) or a
    random variable.
  \item
    Formally, a fixed initial value is a special case of a discrete
    distribution having a point mass with probability one at the fixed
    value. Therefore, fixed initial values are covered in our framework
    since we use probability density functions to describe both discrete
    and continuous probability distributions.
  \item
    Mathematically, a probability mass function (for discrete
    distributions) is a probability density with respect to a
    \href{https://en.wikipedia.org/wiki/Counting_measure}{counting
    measure}. We don't have to get sidetracked on to that topic, but it
    is worth noting that there is a proper mathematical justification
    for treating a probability mass function as a type of probability
    density function.
  \item
    It is not important whether to adopt the convention that the Markov
    process model is intialized at time \(t_1\) or at some previous time
    \(t_0\). Here, we follow the choice to use \(t_0\).
  \end{itemize}
\item
  The probability density function
  \(f_{X_n|X_{n-1}}(x_n\given x_{n-1})\) is called the \textbf{one-step
  transition density} of the Markov process.
\item
  In words, the Markov property says that the next step taken by a
  Markov process follows the one-step transition density based on the
  current state, whatever the previous history of the process.
\item
  For a POMP model, the full joint distribution of the latent process is
  entirely specified by the one-step transition densities, as we will
  show below. Therefore, we also call
  \(f_{X_n|X_{n-1}}(x_n\given x_{n-1})\) the \textbf{process model}.
\end{itemize}

\begin{center}\rule{0.5\linewidth}{\linethickness}\end{center}

\begin{center}\rule{0.5\linewidth}{\linethickness}\end{center}

\subsubsection{Question: Write the joint distribution in terms of the
one-step transition
densities}\label{question-write-the-joint-distribution-in-terms-of-the-one-step-transition-densities}

\begin{itemize}
\tightlist
\item
  Use {[}MP1{]} to derive an expression for the joint distribution of a
  Markov process as a product of the one-step transition densities,
\end{itemize}

{[}MP2{]}
\(\eqspace f_{X_{0:N}}(x_{0:N}) = f_{X_0}(x_0)\prod_{n=1}^N f_{X_n|X_{n-1}}(x_n\given x_{n-1})\).

\begin{center}\rule{0.5\linewidth}{\linethickness}\end{center}

\begin{center}\rule{0.5\linewidth}{\linethickness}\end{center}

\subsubsection{Question: Show that a causal Gaussian AR(1) process is a
Markov
process.}\label{question-show-that-a-causal-gaussian-ar1-process-is-a-markov-process.}

\begin{center}\rule{0.5\linewidth}{\linethickness}\end{center}

\begin{center}\rule{0.5\linewidth}{\linethickness}\end{center}

\subsubsection{Time homogeneous transitions and
stationarity}\label{time-homogeneous-transitions-and-stationarity}

\begin{itemize}
\item
  In general, the one step transition probability density in a POMP
  model can depend on \(n\).
\item
  A latent process model \(X_{0:N}\) is \textbf{time-homogeneous} if the
  one step transition probability density does not depend on \(n\),
  i.e., if there is a conditional density \(f(y\given x)\) such that,
  for all \(n\in 1:N\),
  \[  f_{X_n|X_{n-1}}(x_n\given x_{n-1})= f(x_n\given x_{n-1}).\]
\item
  If \(X_{0:N}\) is stationary then it is time homogeneous.

  \begin{itemize}
  \tightlist
  \item
    Why? Go back to the definitions and show this from first principles.
  \end{itemize}
\item
  Time homogeneity does not necessarily imply stationarity.

  \begin{itemize}
  \item
    Find a counter-example.
  \item
    What has to be added to time homogeneity to get stationarity?
  \end{itemize}
\end{itemize}

\begin{center}\rule{0.5\linewidth}{\linethickness}\end{center}

\begin{center}\rule{0.5\linewidth}{\linethickness}\end{center}

\subsubsection{The measurement model}\label{the-measurement-model}

\begin{itemize}
\item
  We model the observation process random variables \(Y_{1:N}\).
\item
  For state space models, we will generally write the data as
  \(\data{y_{1:N}}\).
\item
  We model the measurement at time \(t_n\) to depend only on the value
  of the latent process at time \(t_n\), conditionally independent of
  all other latent process and observation process variables. Formally,
  this assumption is,
\end{itemize}

{[}MP3{]}
\(\eqspace f_{Y_n|X_{0:N},Y_{1:n-1},Y_{n+1:N}}(y_n\given x_{0:N},y_{1:n-1},y_{n+1:N}) = f_{Y_n|X_n}(y_n\given x_{n})\).

\begin{itemize}
\item
  We call \(f_{Y_n|X_n}(y_n\given x_{n})\) the \textbf{measurement
  model}.
\item
  In general, the measurement model can depend on \(n\).
\item
  The measurement model is \textbf{time-homogeneous} if there is a
  conditional probability density function \(g(y\given x)\) such that,
  for all \(n\in 1:N\),
  \[ f_{Y_n|X_n}(y_n\given x_{n})= g(y_n\given x_n).\]
\end{itemize}

\begin{center}\rule{0.5\linewidth}{\linethickness}\end{center}

\begin{center}\rule{0.5\linewidth}{\linethickness}\end{center}

\subsection{Four basic calculations for working with POMP
models}\label{four-basic-calculations-for-working-with-pomp-models}

\begin{itemize}
\item
  Many time series models in science, engineering and industry can be
  written as POMP models.
\item
  A reason that POMP models form a useful tool for statistical work is
  that there are convenient recursive formulas to carry out following
  four basic calculations.
\end{itemize}

\begin{center}\rule{0.5\linewidth}{\linethickness}\end{center}

\begin{center}\rule{0.5\linewidth}{\linethickness}\end{center}

\subsubsection{Prediction}\label{prediction}

\begin{itemize}
\item
  One-step prediction of the latent process at time \(t_{n+1}\) given
  data up to time \(t_n\) involves finding
  \[ f_{X_{n+1}|Y_{1:n}}(x_{n+1}\given \data{y_{1:n}}).\]
\item
  We may want to carry out prediction (also called forecasting) more
  than one time step ahead. However, unless specified otherwise, the
  prediction calculation will be one-step prediction.
\item
  One-step prediction turns out to be closely related to computing the
  likelihood function, and therefore central to statistical inference.
\item
  We have required our prediction to be a conditional probability
  density, not a point estimate. In the context of forecasting, this is
  called a \textbf{probabilistic forecast}, and has advantages over a
  point estimate forecast. What are they? Are there any disadvantages to
  probabilistic forecasting?
\end{itemize}

\begin{center}\rule{0.5\linewidth}{\linethickness}\end{center}

\begin{center}\rule{0.5\linewidth}{\linethickness}\end{center}

\subsubsection{Filtering}\label{filtering}

\begin{itemize}
\item
  The filtering calculation at time \(t_n\) is to find the conditional
  distribution of the latent process \(X_n\) given currently available
  data, \(\data{y_{1:n}}\).
\item
  Filtering therefore involves calculating
  \[f_{X_{n}|Y_{1:n}}(x_n\given \data{y_{1:n}}).\]
\end{itemize}

\begin{center}\rule{0.5\linewidth}{\linethickness}\end{center}

\begin{center}\rule{0.5\linewidth}{\linethickness}\end{center}

\subsubsection{Smoothing}\label{smoothing}

\begin{itemize}
\item
  In the context of a POMP model, smoothing involves finding the
  conditional distribution of \(X_n\) given all the data,
  \(\data{y_{1:N}}\).
\item
  So, the smoothing calculation is
  \[f_{X_{n}|Y_{1:N}}(x_n\given \data{y_{1:N}}).\]
\end{itemize}

\begin{center}\rule{0.5\linewidth}{\linethickness}\end{center}

\begin{center}\rule{0.5\linewidth}{\linethickness}\end{center}

\subsubsection{The likelihood}\label{the-likelihood}

\begin{itemize}
\item
  The model may depend on a parameter vector \(\theta\).
\item
  Since we haven't explicitly written this dependence above, the
  likelihood calculation is to evaluate the joint density of \(Y_{1:N}\)
  at the data, \[f_{Y_{1:N}}(\data{y_{1:N}}).\]
\item
  If we can compute this for any value of \(\theta\), we can perform
  numerical optimization to get a maximum likelihood estimate, compute
  profile likelihood confidence intervals, carry out likelihood ratio
  tests, and make AIC comparisons.
\end{itemize}

\begin{center}\rule{0.5\linewidth}{\linethickness}\end{center}

\begin{center}\rule{0.5\linewidth}{\linethickness}\end{center}

\subsubsection{The prediction and filtering
formulas}\label{the-prediction-and-filtering-formulas}

\begin{itemize}
\tightlist
\item
  One-step prediction of the latent process at time \(t_{n}\) given data
  up to time \(t_{n-1}\) can be computed in terms of the filtering
  problem at time \(t_{n-1}\), via the \textbf{prediction formula} for
  \(n\in 1:N\),
\end{itemize}

{[}MP4{]}
\(\eqspace \displaystyle f_{X_{n}|Y_{1:n-1}}(x_{n}\given \data{y_{1:n-1}})\)

\(\eqspace \eqspace \displaystyle = \int f_{X_{n-1}|Y_{1:n-1}}(x_{n-1}\given \data{y_{1:n-1}}) f_{X_{n}|X_{n-1}}(x_{n}\given x_{n-1})\, dx_{n-1}\).

\begin{itemize}
\item
  To make this formula work for \(n=1\), we need the convention that
  \(1:k\) is the empty set when \(k=0\). Conditioning on an empty
  collection of random variables is the same as not conditioning at all!
  In this case, we have by definition that
  \[ f_{X_{0}|Y_{1:0}}(x_{0}\given \data{y_{1:0}}) = f_{X_0}(x_0).\] In
  other words, the filtering calcuation at time \(t_0\) is the initial
  density for the latent process. This makes sense, since at time
  \(t_0\) we have no data to condition on.
\item
  To see why the prediction formula is true, we can view it is an
  application of a general identity for joint continuous random
  variables \(X\), \(Y\), \(Z\),
  \[ f_{X|Y}(x\given y)= \int f_{XZ|Y}(x,z\given y)\, dz,\] which is a
  condition form of the basic identity that integrating out a joint
  distribution gives a marginal distribution.
\end{itemize}

\begin{center}\rule{0.5\linewidth}{\linethickness}\end{center}

\begin{center}\rule{0.5\linewidth}{\linethickness}\end{center}

\begin{itemize}
\tightlist
\item
  Filtering at time \(t_n\) can be computed by combining the new
  information in the datapoint \(\data{y_{n}}\) with the calculation of
  the one-step prediction of the latent process at time \(t_{n}\) given
  data up to time \(t_{n-1}\). This is carried out via the
  \textbf{filtering formula} for \(n\in 1:N\),
\end{itemize}

{[}MP5{]}
\(\eqspace \displaystyle f_{X_{n}|Y_{1:n}}(x_{n}\given \data{y_{1:n}}) = \frac{ f_{X_{n}|Y_{1:n-1}}(x_{n}\given \data{y_{1:n-1}})\, f_{Y_n|X_n}(\data{y_n}\given x_n) }{ f_{Y_{n}|Y_{1:n-1}}(\data{y_n}\given \data{y_{1:n-1}}) }\).

\begin{itemize}
\tightlist
\item
  The denominator in the filtering formula {[}MP5{]} is the
  \textbf{conditional likelihood} of \(\data{y_{n}}\) given
  \(\data{y_{1:n-1}}\). It can be computed in terms of the one-step
  prediction density, via the \textbf{conditional likelihood formula},
\end{itemize}

{[}MP6{]}
\(\eqspace \displaystyle f_{Y_{n}|Y_{1:n-1}}(\data{y_n}\given \data{y_{1:n-1}}) = \int f_{X_{n}|Y_{1:n-1}}(x_{n}\given \data{y_{1:n-1}})\, f_{Y_n|X_n}(\data{y_n}\given x_n)\, dx_n\).

\begin{itemize}
\item
  To make this formula work for \(n=1\), we again take advantage of the
  convention that \(1:k\) is the empty set when \(k=0\).
\item
  To see why the filtering formula is true, we can apply the identity
  for joint continuous random variables \(X\), \(Y\), \(Z\),
  \[f_{X|YZ}(x\given y,z)= \frac{ f_{Y|XZ}(y\given x,z)\, f_{X|Z}(x\given z)}{f_{Y|Z}(y\given z)}.\]
  This is a conditional form of the Bayes formula, which in turn is
  closely related to a conditional form of the basic definition of the
  conditional probability density function,
  \[f_{X|YZ}(x\given y,z)= \frac{f_{XY|Z}(x,y\given z)}{f_{Y|Z}(y\given z)}.\]
\end{itemize}

\begin{center}\rule{0.5\linewidth}{\linethickness}\end{center}

\begin{center}\rule{0.5\linewidth}{\linethickness}\end{center}

\begin{itemize}
\tightlist
\item
  The prediction and filtering formulas are \textbf{recursive}. If they
  can be computed for time \(t_n\) then they provide the foundation for
  the following computation at time \(t_{n+1}\).
\end{itemize}

\begin{center}\rule{0.5\linewidth}{\linethickness}\end{center}

\begin{center}\rule{0.5\linewidth}{\linethickness}\end{center}

\subsubsection{Question: Give a detailed derivation of {[}MP4{]},
{[}MP5{]} and {[}MP6{]}, being careful to note when you use the Markov
property
{[}MP1{]}.}\label{question-give-a-detailed-derivation-of-mp4-mp5-and-mp6-being-careful-to-note-when-you-use-the-markov-property-mp1.}

\begin{center}\rule{0.5\linewidth}{\linethickness}\end{center}

\begin{center}\rule{0.5\linewidth}{\linethickness}\end{center}

\subsubsection{Computation of the
likelihood}\label{computation-of-the-likelihood}

\begin{itemize}
\tightlist
\item
  The likelihood of the entire dataset, \(\data{y_{1:N}}\) can be found
  from {[}MP6{]}, using the identity
\end{itemize}

{[}MP7{]}
\(\eqspace\displaystyle f_{Y_{1:N}}(\data{y_{1:N}}) = \prod_{n=1}^N f_{Y_{n}|Y_{1:n-1}}(\data{y_n}\given \data{y_{1:n-1}})\).

\begin{itemize}
\item
  Yet again, this formula {[}MP7{]} requires the convention that \(1:k\)
  is the empty set when \(k=0\), so the first term in the product is
  \[f_{Y_{1}|Y_{1:0}}(\data{y_1}\given \data{y_{1:0}}) = 
  f_{Y_{1}}(\data{y_1}).\]
\item
  If our model has an unknown parameter \(\theta\), the likelihood
  identity {[}MP7{]} lets us evaluate the log likelihood function,
  \[\loglik(\theta)=\log f_{Y_{1:N}}(\data{y_{1:N}}\params\theta).\]
\end{itemize}

\begin{center}\rule{0.5\linewidth}{\linethickness}\end{center}

\begin{center}\rule{0.5\linewidth}{\linethickness}\end{center}

\subsubsection{The smoothing formulas}\label{the-smoothing-formulas}

\begin{itemize}
\item
  Smoothing is less fundamental for likelihood-based inference than
  filtering and one-step prediction.
\item
  Nevertheless, sometimes we want to compute the smoothing density, so
  let's obtain some necessary formulas.
\item
  The filtering and prediction formulas are recursions forwards in time
  (we use the solution at time \(t_{n-1}\) to carry out the computation
  at time \(t_{n}\)).
\item
  There are similar \textbf{backwards recursion formulas},
\end{itemize}

{[}MP8{]}
\(\eqspace f_{Y_{n:N}|X_{n}}(\data{y_{n:N}}\given x_n)= f_{Y_n|X_n}(\data{y_n}\given x_n) f_{Y_{n+1:N}|X_{n}}(\data{y_{n+1:N}}\given x_n)\).

{[}MP9{]}
\(\eqspace \displaystyle f_{Y_{n+1:N}|X_{n}}(\data{y_{n+1:N}}\given x_n)\)
\(\eqspace\eqspace\displaystyle = \int f_{Y_{n+1:N}|X_{n+1}}(\data{y_{n+1:N}}\given x_{n+1}) \, f_{X_{n+1}|X_n}(x_{n+1}\given x_n)\, dx_{n+1}\).

\begin{itemize}
\tightlist
\item
  The forwards and backwards recursion formulas together allow us to
  compute the \textbf{smoothing formula},
\end{itemize}

{[}MP10{]}
\(\myeq{ f_{X_{n}|Y_{1:N}}(x_n\given \data{y_{1:N}}) = \frac{ f_{X_{n}|Y_{1:n-1}}(x_{n}\given \data{y_{1:n-1}}) \, f_{Y_{n:N}|X_{n}}(\data{y_{n:N}}\given x_n) }{ f_{Y_{n:N}|Y_{1:n-1}}(\data{y_{n:N}}\given \data{y_{1:n-1}}) } }\).

\subsubsection{Question: Show how {[}MP8{]}, {[}MP9{]} and {[}MP10{]}
follow from the basic properties of conditional densities combined with
the Markov
property.}\label{question-show-how-mp8-mp9-and-mp10-follow-from-the-basic-properties-of-conditional-densities-combined-with-the-markov-property.}

\begin{center}\rule{0.5\linewidth}{\linethickness}\end{center}

\begin{center}\rule{0.5\linewidth}{\linethickness}\end{center}

\subsubsection{Un-normalized filtering and
smoothing}\label{un-normalized-filtering-and-smoothing}

\begin{itemize}
\item
  Some common Monte Carlo algorithms (Markov chain Monte Carlo and
  self-normalized importance sampling) need probability density
  functions only up to an unknown constant factor. These algorithms
  depend on ratios of densities, for which a constant factor cancels out
  and so does not have to be computed.
\item
  In some analytic and numeric computations, it is helpful to avoid
  calculating a normalizing constant for a density, since it can be
  worked out later using the property that the probability density
  function must integrate to 1.
\item
  The denominators
  \(f_{Y_{n}|Y_{1:n-1}}(\data{y_n}\given \data{y_{1:n-1}})\) and
  \(f_{Y_{n:N}|Y_{1:n-1}}(\data{y_{n:N}}\given \data{y_{1:n-1}})\), in
  equations {[}MP5{]} and {[}MP10{]} respectively, may sometimes be hard
  to compute.
\item
  When we are only interested in computing the filtering and smoothing
  densities up to an unknown constant, we can simplify {[}MP5{]} and
  {[}MP10{]} using the proportionality relationship \(\propto\). This
  gives,
\end{itemize}

{[}MP5\(^\prime\){]}
\(\myeq{ f_{X_{n}|Y_{1:n}}(x_{n}\given \data{y_{1:n}}) \propto f_{X_{n}|Y_{1:n-1}}(x_{n}\given \data{y_{1:n-1}})\, f_{Y_n|X_n}(\data{y_n}\given x_n) }\),

{[}MP10\(^\prime\){]}
\(\myeq{ f_{X_{n}|Y_{1:N}}(x_n\given \data{y_{1:N}}) \propto f_{X_{n}|Y_{1:n-1}}(x_{n}\given \data{y_{1:n-1}}) \, f_{Y_{n:N}|X_{n}}(\data{y_{n:N}}\given x_n) }\).

\begin{itemize}
\tightlist
\item
  Note that the normalizing ``constant'' in equations {[}MP5{]} and
  {[}MP10{]} does depend on \(\data{y_{1:N}}\). However, the data are
  fixed constant values. The variable in both these equations is
  \(x_n\).
\end{itemize}

\subsection{Linear Gaussian POMP (LG-POMP)
models}\label{linear-gaussian-pomp-lg-pomp-models}

\begin{itemize}
\item
  Linear Gaussian partially observed Markov process (LG-POMP) models
  have many applications

  \begin{itemize}
  \item
    Gassian ARMA models are LG-POMP models. The POMP recursion formulas
    give a computationally efficient way to obtain the likelihood of a
    Gaussian ARMA model.
  \item
    The computations for smoothing splines can be written as an LG-POMP
    model. This gives a route to computationally efficient spline
    smooothing.
  \item
    The so-called \textbf{Basic Structural Model} is an LG-POMP used for
    econometric forecasting. It models a stochastic trend, seasonality,
    and measurement error, in a framework with econometrically
    interpretable parameters. By contrast, all but the simplest ARMA
    models are usually \textbf{black box} methods, which may fit the
    data but are not readily interpretable.
  \item
    LG-POMP models are widely used in engineering, especially for
    control applications.
  \item
    In all scientific and engineering applications, if your situation is
    not too far from linear and Gaussian, you save a lot of effort if an
    LG-POMP model is appropriate. General nonlinear POMP models usually
    involve intensive Monte Carlo computation.
  \end{itemize}
\end{itemize}

\begin{center}\rule{0.5\linewidth}{\linethickness}\end{center}

\begin{center}\rule{0.5\linewidth}{\linethickness}\end{center}

\subsubsection{The general LG-POMP
model}\label{the-general-lg-pomp-model}

\begin{itemize}
\item
  Suppose the latent process, \(X_{0:N}\), takes vector values with
  dimension \(d_X\).
\item
  Suppose the observation process, \(\{Y_n\}\), takes vector values with
  dimension \(d_Y\).
\item
  A general mean zero LG-POMP model is specified by

  \begin{itemize}
  \item
    A sequence of \(d_X\times d_X\) matrices, \(\matA_{1:N}\),
  \item
    A sequence of \(d_X\times d_X\) covariance matrices,
    \(\covmatX_{0:N}\),
  \item
    A sequence of \(d_Y\times d_X\) matrices, \(\matB_{1:N}\)
  \item
    A sequence of \(d_Y\times d_Y\) covariance matrices,
    \(\covmatY_{1:N}\).
  \end{itemize}
\item
  We initialize with \(X_0\sim N[0,\covmatX_0]\) and then define the
  entire LG-POMP model by a recursion for \(n\in 1:N\),
\end{itemize}

{[}LG1{]} \(\eqspace X_{n} = \matA_n X_{n-1} + \epsilon_n\),
\(\eqspace \epsilon_n\sim N[0,\covmatX_n]\),

{[}LG2{]} \(\eqspace Y_{n} = \matB_n X_n + \eta_n\),
\(\eqspace \eta_n\sim N[0,\covmatY_n]\).

\begin{itemize}
\tightlist
\item
  Often, but not always, we will have a \textbf{time-homogeneous}
  LG-POMP model, with \(\matA_n=\matA\), \(\;\matB_n=\matB\),
  \(\;\covmatX_n=\covmatX\) and \(\covmatY_n=\covmatY\) for
  \(n\in 1:N\).
\end{itemize}

\begin{center}\rule{0.5\linewidth}{\linethickness}\end{center}

\begin{center}\rule{0.5\linewidth}{\linethickness}\end{center}

\subsection{The LG-POMP representation of a Gaussian
ARMA}\label{the-lg-pomp-representation-of-a-gaussian-arma}

\begin{itemize}
\tightlist
\item
  Suppose \(\{Y_n\}\) is a Gaussian ARMA(p,q) model with noise process
  \(\omega_n\sim N[0,\sigma^2]\) and specification
\end{itemize}

{[}LG3{]}
\(\eqspace\displaystyle Y_n = \sum_{j=1}^p \ar_j Y_{n-j} + \omega_n + \sum_{k=1}^q \ma_q \omega_{n-k}.\)

\begin{itemize}
\item
  Set \(r=\max(p,q+1)\) so that \(\{Y_n\}\) is also ARMA(r,r-1).
\item
  Our LG-POMP representation has \(d_X=r\), with
  \[\matB_n = \matB = (1,0,0,\dots,0)\] and
  \[\covmatY_n = \covmatY = 0.\]
\item
  Therefore, \(Y_n\) is the first component of \(X_n\), observed without
  measurement error.
\item
  Now, define \[ X_n = \left(\begin{array}{l}
  Y_n \\
  \ar_2 Y_{n-1} + \dots + \ar_r Y_{n-r+1} + \ma_1 \omega_n + \dots +\ma_{r-1} \omega_{n-r+2}
  \\
  \ar_3 Y_{n-1} + \dots + \ar_r Y_{n-r+1} + \ma_2 \omega_n + \dots +\ma_{r-1} \omega_{n-r+3}
  \\
  \vdots
  \\
  \ar_r Y_{n-1} + \ma_{r-1} \omega_t
  \end{array}\right)\]
\item
  We can check that the ARMA equation {[}LG3{]} corresponds to the
  matrix equation \[ X_{n} = \matA X_{n-1} + \left(\begin{array}{l}
  1 \\
  \ma_1\\
  \ma_2\\
  \vdots\\
  \ma_{r-1}
  \end{array}\right) \omega_n.
  \] where \[
  \matA = 
  \left(\begin{array}{ccccc}
  \ar_1 & 1 & 0 & \ldots & 0 \\
  \ar_2 & 0 & 1 &       & 0 \\
  \vdots & \vdots& \ddots & \\
  \ar_{r-1} & 0 & & 1 \\
  \ar_{r} & 0 \ldots & 0
  \end{array}\right).\] This is in the form of a time-homogenous
  LG-POMP, with \(\matA\), \(\matB\) and \(\covmatY\) defined above, and
  \[\covmatX_n = \covmatX = \sigma^2 (1, \ma_1,\ma_2,\dots,\ma_{r-1})^{\transpose}(1, \ma_1,\ma_2,\dots,\ma_{r-1}).\]
\item
  There are other LG-POMP representations giving rise to the same ARMA
  model.
\item
  When only one component of a latent process is observed, any model
  giving rise to the same observed component is indistinguishable from
  the data.
\item
  Here, the LG-POMP model has order \(r^2\) parameters and the ARMA
  model has order \(r\) parameters, so we might expect there are many
  ways to parameterize the ARMA model as a special case of the much
  larger LG-POMP model.
\end{itemize}

\begin{center}\rule{0.5\linewidth}{\linethickness}\end{center}

\begin{center}\rule{0.5\linewidth}{\linethickness}\end{center}

\subsection{The basic structural model and its LG-POMP
representation}\label{the-basic-structural-model-and-its-lg-pomp-representation}

\begin{itemize}
\item
  The \textbf{basic structural model} is an econometric model used for
  forecasting.
\item
  The basic stuctural model supposes that the observation process
  \(Y_{1:N}\) is the sum of a \textbf{level} (\(L_n\)), a \textbf{trend}
  (\(T_n\)) describing the rate of change of the level, and a monthly
  \textbf{seasonal component} (\(S_n\)). The model supposes that all
  these quantities are perturbed with Gaussian white noise at each time
  point. So, we have the following model equations
\end{itemize}

\(\begin{array}{lrcl} \mbox{[BSM1]} \eqspace & Y_n &=& L_n + S_n + \epsilon_n \\ \mbox{[BSM2]} \eqspace & L_{n} &=& L_{n-1} + T_{n-1} + \xi_n \\ \mbox{[BSM3]} \eqspace & T_{n} &=& T_{n-1} + \zeta_n \\ \mbox{[BSM4]} \eqspace & S_{n} &=& -\sum_{k=1}^{11} S_{n-k} + \eta_n \end{array}\)

\begin{itemize}
\item
  We suppose \(\epsilon_n\sim N[0,\sigma^2_\epsilon]\),
  \(\xi_n\sim N[0,\sigma^2_\xi]\), \(\zeta_n\sim N[0,\sigma^2_\zeta]\),
  and \(\eta_n\sim N[0,\sigma^2_\eta]\).
\item
  The \textbf{local linear trend} model is the basic structural model
  without the seasonal component, \(\{S_n\}\)
\item
  The \textbf{local level model} is the basic structural model without
  either the seasonal component, \(\{S_n\}\), or the trend component,
  \(\{T_n\}\). The local level model is therefore a random walk observed
  with measurement error.
\item
  To complete the model, we need to specify initial values. This is not
  fully explained in the documentation of the R implementation of the
  basic structural model, \texttt{StructTS}. We could go through the
  source code to find out what it does.
\item
  Incidentally, \texttt{?StructTS} does give some advice which resonates
  with our experience earlier in the course that optimization for ARMA
  models is often imperfect.

  ``Optimization of structural models is a lot harder than many of the
  references admit. For example, the
  âÂ\euro{}˜AirPassengersâÂ\euro{}™ data are considered in Brockwell
  \& Davis (1996): their solution appears to be a local maximum, but
  nowhere near as good a fit as that produced by
  âÂ\euro{}˜StructTSâÂ\euro{}™. It is quite common to find fits with
  one or more variances zero, and this can include sigma\^{}2\_eps.''
\item
  To put {[}BSM1-4{]} in the form of an LG-POMP model, we set
\end{itemize}

{[}BSM5{]}
\(\eqspace X_n = (L_n,T_n,S_n, S_{n-1}, S_{n-2}, \dots,S_{n-10})^{\transpose}\).

\begin{itemize}
\tightlist
\item
  Then, we have
\end{itemize}

{[}BSM6{]} \(\eqspace Y_n = (1,0,1,0,0,\dots, 0) X_n + \epsilon_n\),

\(\displaystyle  \left(\begin{array}{l} L_{n} \\ T_{n} \\ S_{n} \\ S_{n-1}\\ S_{n-2} \\  \vdots \\ S_{n-10} \end{array}\right) = \left(\begin{array}{ccccccc} 1 & 1 & 0 & 0 & 0 & \ldots & 0 \\ 0 & 1 & 0 & 0 & 0 & \ldots & 0 \\ 0 & 0 & -1 & -1 & -1 & \ldots & -1\\ 0 & 0 & 1 & 0 & 0 & \ldots & 0 \\ 0 & 0 & 0 & 1 & 0 & \ldots & 0 \\ \vdots & & &\ddots & \ddots & \ddots & \vdots \\ 0 & 0 & 0 &\ldots & 0 & 1 & 0 \end{array}\right) \left(\begin{array}{l} L_{n-1} \\ T_{n-1} \\ S_{n-1} \\ S_{n-2}\\ S_{n-3} \\  \vdots \\ S_{n-11} \end{array}\right) + \left(\begin{array}{l} \xi_n \\ \zeta_n \\ \eta_n \\ 0 \\ 0 \\  \vdots \\ 0 \end{array}\right).\)

\begin{itemize}
\tightlist
\item
  From {[}BSM5{]} and {[}BSM6{]}, we can read off the matrices
  \(\matA\), \(\matB\), \(\covmatX\) and \(\covmatY\) in the LG-POMP
  representation of the basic structural model.
\end{itemize}

\begin{center}\rule{0.5\linewidth}{\linethickness}\end{center}

\begin{center}\rule{0.5\linewidth}{\linethickness}\end{center}

\subsection{Spline smoothing and its LG-POMP
representation}\label{spline-smoothing-and-its-lg-pomp-representation}

\begin{itemize}
\item
  Spline smoothing is a standard method to smooth scatter plots and time
  plots.
\item
  For example, \texttt{smooth.spline} in R.
\item
  A \textbf{smoothing spline} for an equally spaced time series
  \(\data{y_{1:N}}\) collected at times \(t_{1:N}\) is the sequence
  \(x_{1:N}\) minimizing the \textbf{penalized sum of squares (PSS)},
  which is defined as
\end{itemize}

{[}SS1{]}
\(\myeq{  \mathrm{PSS}(x_{1:N}\params\lambda)  = \sum_{n=1}^N(\data{y_n}-x_n)^2 + \lambda\sum_{n=3}^N(\Delta^2 x_n)^2 }\).

\begin{itemize}
\item
  The spline is defined for all times, but here we are only concerned
  with its value at the times \(t_{1:N}\).
\item
  Here, \(\Delta x_n = (1-B)x_n = x_n - x_{n-1}.\)
\item
  The \textbf{smoothing parameter}, \(\lambda\), penalizes \(x_{1:N}\)
  to prevent the spline from interpolating the data.
\item
  If \(\lambda=0\), the spline will go through each data point, i.e,
  \(x_{1:N}\) will interpolate \(\data{y_{1:N}}\).
\item
  If \(\lambda=\infty\), the spline will be the ordinary least squares
  regression fit, \[ x_n = \alpha + \beta n,\] since
  \(\Delta^2(\alpha + \beta n) = 0\).
\item
  Now consider the model,
\end{itemize}

{[}SS2{]}
\(\eqspace \begin{array}{rclcl} X_n &=& 2X_{n-1}-X_{n-2} + \epsilon_n, & & \epsilon_n\sim \mathrm{iid}\; N[0,\sigma^2/\lambda]\\ Y_n &=& X_n + \eta_n & & \eta_n\sim \mathrm{iid}\; N[0,\sigma^2]. \end{array}\)

\begin{itemize}
\tightlist
\item
  Note that \(\Delta^2 X_n = \epsilon_n\).
\end{itemize}

\begin{center}\rule{0.5\linewidth}{\linethickness}\end{center}

\begin{center}\rule{0.5\linewidth}{\linethickness}\end{center}

\subsubsection{Question: Constructing a linear Gaussian POMP (LG-POMP)
model from
{[}SS2{]}.}\label{question-constructing-a-linear-gaussian-pomp-lg-pomp-model-from-ss2.}

\begin{itemize}
\tightlist
\item
  Note that \(\{X_n,Y_n\}\) defined in {[}SS2{]} is not quite an LG-POMP
  model. However, we can use \(\{X_n\}\) and \(\{Y_n\}\) to build an
  LG-POMP model. How?
\end{itemize}

\begin{center}\rule{0.5\linewidth}{\linethickness}\end{center}

\begin{center}\rule{0.5\linewidth}{\linethickness}\end{center}

\begin{itemize}
\item
  The joint density of \(X_{1:N}\) and \(Y_{1:N}\) in {[}SS2{]} can be
  written as
  \[ f_{X_{1:N}Y_{1:N}}(x_{1:N},y_{1:N})= f_{X_{1:N}}(x_{1:N}) \, f_{Y_{1:N}|X_{1:N}}(y_{1:N}\given x_{1:N}).\]
\item
  Taking logs,
  \[ \log f_{X_{1:N}Y_{1:N}}(x_{1:N},y_{1:N})= \log f_{X_{1:N}}(x_{1:N}) + \log f_{Y_{1:N}|X_{1:N}}(y_{1:N}\given x_{1:N}).\]
\item
  Suppose that initial conditions are irrelevant (they could be either
  unknown parameters or an improper Gaussian distribution with infinite
  variance). Then, noting that \(\{\Delta^2 X_{n}, n\in 1:N\}\) and
  \(\{Y_n-X_n, n\in 1:N\}\) are collections of independent Normal random
  variables with mean zero and variances \(\sigma^2/\lambda\) and
  \(\sigma^2\) respectively, we have
\end{itemize}

{[}SS3{]}
\(\myeq{ \log f_{X_{1:N}Y_{1:N}}(x_{1:N},y_{1:N} \params \sigma,\lambda) = \frac{-1}{2\sigma^2} \sum_{n=1}^N(y_n-x_n)^2 +\frac{-\lambda}{2\sigma^2} \sum_{n=3}^N(\Delta^2 x_n)^2 + C }\).

\begin{itemize}
\item
  In {[}SS3{]}, \(C\) is a constant that depends on \(\sigma\) and
  \(\lambda\) but not on \(x_{1:N}\) or \(y_{1:N}\).
\item
  Comparing {[}SS3{]} with {[}SS1{]}, we see that maximizing the density
  \(f_{X_{1:N}Y_{1:N}}(x_{1:N},\data{y_{1:N}} \params \sigma,\lambda)\)
  as a function of \(x_{1:N}\) is the same problem as finding the
  smoothing spline by minimizing the penalized sum of squares in
  {[}SS1{]}.
\item
  For a Gaussian density, the mode (i.e., the maximum of the density) is
  equal to the expected value. Therefore, we have
\end{itemize}

\begin{eqnarray}
\arg\min_{x_{1:N}} \mathrm{PSS}(x_{1:N}\params\lambda),
&=& 
\arg\max_{x_{1:N}} f_{X_{1:N}Y_{1:N}}(x_{1:N},\data{y_{1:N}}\params \sigma,\lambda),
\\
&=&
\arg\max_{x_{1:N}} \frac{
  f_{X_{1:N}Y_{1:N}}(x_{1:N},\data{y_{1:N}} \params \sigma,\lambda)
}{
  f_{Y_{1:N}}(\data{y_{1:N}} \params \sigma,\lambda)
},
\\
&=& 
\arg\max_{x_{1:N}} f_{X_{1:N}|Y_{1:N}}(x_{1:N}\given \data{y_{1:N}}\params \sigma,\lambda),
\\
&=& \E\big[X_{1:N}\given Y_{1:N}=\data{y_{1:N}} \params  \sigma,\lambda\big].
\end{eqnarray}

\begin{itemize}
\item
  The smoothing calculation for an LG-POMP model involves finding the
  mean and variance of \(X_{n}\) given \(Y_{1:N}=\data{y_{1:N}}\).
\item
  We conclude that the smoothing problem for this LG-POMP model is the
  same as the spline smoothing problem defined by {[}SS1{]}.
\item
  If you have experience using smoothing splines, this connection may
  help you transfer that experience to POMP models.
\item
  Once you have experience with POMP models, this connection helps you
  understand smoothers that are commonly used in many applications.
\item
  For example, we might propose that the smoothing parameter \(\lambda\)
  could be selected by maximum likelihood for the POMP model.
\end{itemize}

\begin{center}\rule{0.5\linewidth}{\linethickness}\end{center}

\begin{center}\rule{0.5\linewidth}{\linethickness}\end{center}

\subsubsection{\texorpdfstring{Question: Why do we use
\(\Delta^2 X_n=\epsilon_n\) for our smoothing
model?}{Question: Why do we use \textbackslash{}Delta\^{}2 X\_n=\textbackslash{}epsilon\_n for our smoothing model?}}\label{question-why-do-we-use-delta2-x_nepsilon_n-for-our-smoothing-model}

\begin{itemize}
\item
  Seeing that the smoothing spline arrives from the particular choice of
  LG-POMP model in equation {[}SS2{]} could make you wonder why we
  choose that model.
\item
  Any ideas?
\item
  Even if this LG-POMP model is sometimes reasonable, presumably there
  are other occasions when a different LG-POMP model would be a superior
  choice for smoothing.
\end{itemize}

\begin{center}\rule{0.5\linewidth}{\linethickness}\end{center}

\begin{center}\rule{0.5\linewidth}{\linethickness}\end{center}

\subsection{The Kalman filter}\label{the-kalman-filter}

\begin{itemize}
\item
  We find exact versions of the prediction, filtering and smoothing
  formulas {[}MP4--10{]} for the linear Gaussian partially observed
  Markov process (LG-POMP) model {[}LG1,LG2{]}.
\item
  In the linear Gaussian case, the conditional probability density
  functions in {[}MP4--10{]} are specified by the conditional mean and
  conditional variance.
\end{itemize}

\begin{center}\rule{0.5\linewidth}{\linethickness}\end{center}

\begin{center}\rule{0.5\linewidth}{\linethickness}\end{center}

\subsubsection{Review of the multivariate normal
distribution}\label{review-of-the-multivariate-normal-distribution}

\begin{itemize}
\item
  A random variable \(X\) taking values in \(\R^{d_X}\) is
  \textbf{multivariate normal} with mean \(\mu_X\) and variance
  \(\Sigma_X\) if we can write \[X = \matH Z + \mu_X,\] where \(Z\) is a
  vector of \(d_X\) independent identically distributed \(N[0,1]\)
  random variables and \(\matH\) is a \(d_X\times d_X\) matrix square
  root of \(\Sigma_X\), i.e., \[\matH\matH^{\transpose} = \Sigma_X.\]
\item
  The choice of \(\matH\) is not unique, and a matrix square root of
  this type exists for any covariance matrix. Mathematically, this is
  true because covariance matrices are
  \href{https://en.wikipedia.org/wiki/Positive-definite_matrix\#Positive-semidefinite}{positive
  semi-definite}.
\item
  We write \(X\sim N\big[\mu_X,\Sigma_X\big]\).
\item
  \(X\sim N\big[\mu_X,\Sigma_X\big]\) has a probability density function
  if and only if \(\Sigma_X\) is invertible. This density is given by
  \[f_X(x) = \frac{1}{(2\pi)^{d_X/2}|\Sigma_X|}
  \exp
    \left\{-
   \frac{ (x - \mu_X)\, \big[\Sigma_X\big]^{-1} \, (x - \mu_X)^{\transpose}}{2}
    \right\}.
  \]
\item
  \(X\) and \(Y\) are \textbf{jointly multivariate normal} if the
  combined vector \[W=\left(\begin{array}{l}
  X \\
  Y
  \end{array}\right)\] is multivariate normal. In this case, we write \[
  \mu_W = \left(\begin{array}{l}
  \mu_X \\
  \mu_Y
  \end{array}\right),
  \eqspace
  \Sigma_W = \left(\begin{array}{cc}
  \Sigma_X & \Sigma_{XY}\\
  \Sigma_{YX} & \Sigma_Y
  \end{array}\right),\] where
  \[\Sigma_{XY}= \cov(X,Y) = \E\big[(X-\mu_X)(Y-\mu_Y)^{\transpose}\big].\]
\item
  For jointly multivariate normal random variables \(X\) and \(Y\), we
  have the useful property that the conditional distribution of \(X\)
  given \(Y=y\) is multivariate normal, with conditional mean and
  variance
\end{itemize}

{[}KF1{]}

\begin{eqnarray}
\mu_{X|Y}(y) &=& \mu_X + \Sigma_{XY}\Sigma_Y^{-1}\big(y-\mu_Y\big),
\\
\Sigma_{X|Y} &=& \Sigma_X - \Sigma_{XY}\Sigma_Y^{-1}\Sigma_{YX}.
\end{eqnarray}

\begin{itemize}
\item
  We write this as
  \[ X\given Y=y \sim N\big[\mu_{X|Y}(y),\Sigma_{X|Y}\big]. \]
\item
  In general, the conditional variance of \(X\) given \(Y=y\) will
  depend on \(y\) (remind yourself of the definition of conditional
  variance). In the special case where \(X\) and \(Y\) are jointly
  multivariate normal, this conditional variance happens not to depend
  on the value of \(y\).
\item
  If \(\Sigma_Y\) is not invertible, to make {[}KF1{]} work we have to
  interpret \(\Sigma_Y^{-1}\) as a
  \href{https://en.wikipedia.org/wiki/Generalized_inverse}{generalized
  inverse}.
\end{itemize}

\begin{center}\rule{0.5\linewidth}{\linethickness}\end{center}

\begin{center}\rule{0.5\linewidth}{\linethickness}\end{center}

\begin{itemize}
\tightlist
\item
  To write the Kalman filter, we define the following notation,
\end{itemize}

{[}KF2{]}
\(\eqspace \begin{array}{lcl} X_{n}\given Y_{1:n-1}\equals y_{1:n-1} &\sim& N\big[ \mu_{n}^P(y_{1:n-1}),\Sigma_n^P\big], \\ X_n\given Y_{1:n}\equals y_{1:n} &\sim& N\big[ \mu_n^F(y_{1:n-1}),\Sigma_n^F\big], \\ X_n\given Y_{1:N}\equals y_{1:N} &\sim& N\big[ \mu_n^S(y_{1:N}),\Sigma_n^S\big]. \end{array}\)

\begin{itemize}
\tightlist
\item
  To relate this notation to the general POMP recursion formulas, given
  data \(\data{y_{1:N}}\), we define the following terminology:
\end{itemize}

\(\mu_n^P(\data{y_{1:n-1}})=\E\big[X_n\given Y_{1:n-1}\equals \data{y_{1:n-1}}\big]\)
is the \textbf{one-step prediction mean} for time \(t_n\). It is an
arbitrary decision we have made to call this the prediction for time
\(t_n\) (the time for which the prediction is being made) rather than
for time \(t_{n-1}\) (the time at which the prediction for time \(t_n\)
becomes available).

\(\Sigma_n^P(\data{y_{1:n-1}})=\var\big(X_n\given Y_{1:n-1}\equals \data{y_{1:n-1}}\big)\)
is the \textbf{one-step prediction variance} for time \(t_n\). To make
this terminology work for general POMP models as well as for LG-POMP
models, we have included the dependence on \(\data{y_{1:n-1}}\).

\(\mu_n^F(\data{y_{1:n}})=\E\big[X_n\given Y_{1:n}\equals \data{y_{1:n}}\big]\)
is the \textbf{filter mean} for time \(t_n\).

\(\Sigma_n^F(\data{y_{1:n}})=\var\big(X_n\given Y_{1:n}\equals \data{y_{1:n}}\big)\)
is the \textbf{filter variance} for time \(t_n\).

\(\mu_n^S(\data{y_{1:N}})=\E\big[X_n\given Y_{1:N}\equals \data{y_{1:N}}\big]\)
is the \textbf{smoothing mean} for time \(t_n\).

\(\Sigma_n^S(\data{y_{1:N}})=\var\big(X_n\given Y_{1:N}\equals \data{y_{1:N}}\big)\)
is the \textbf{smoothing variance} for time \(t_n\).

\begin{itemize}
\item
  We have defined the above quantities as estimates rather than
  estimators. For example, we could define the filter mean estimator to
  be the function which is evaluated at the data to give the filter
  mean.
\item
  From the results for linear combinations of Normal random variables,
  we get the Kalman filter and prediction recursions:
\end{itemize}

{[}KF3{]}
\(\myeq{  \mu_{n+1}^P(y_{1:n}) = \matA_{n+1} \mu_{n}^F(y_{1:n}) }\),

{[}KF4{]}
\(\myeq{  \Sigma_{n+1}^P = \matA_{n+1} \Sigma_{n}^F \matA_{n+1}^{\transpose} + \covmatX_{n+1} }\).

{[}KF5{]}
\(\myeq{  \Sigma_{n}^F = \big([\Sigma_n^P]^{-1} + \matB_n^{\transpose}\covmatY_n^{-1}\matB_n\big)^{-1} }\).

{[}KF6{]}
\(\myeq{  \mu_{n}^F(y_{1:n}) = \mu_{n}^P(y_{1:n-1}) + \Sigma_n^F \matB^{\transpose}_n\covmatY_n^{-1}\big\{y_n - \matB_n\mu_n^P(y_{1:n-1})\big\} }\).

\begin{itemize}
\item
  These equations are easy to code, and quick to compute unless the
  dimension of the latent space is very large. In numerical weather
  forecasting, with careful programming, they are solved with latent
  variables having dimension \(d_X\approx 10^7\).
\item
  A similar computation gives backward Kalman recursions. Putting the
  forward and backward Kalman recursions together, as in {[}MP10{]}, is
  called \textbf{Kalman smoothing}.
\end{itemize}

\begin{center}\rule{0.5\linewidth}{\linethickness}\end{center}

\begin{center}\rule{0.5\linewidth}{\linethickness}\end{center}

\subsubsection{Question: Add details to the derivations of
{[}KF3--6{]}}\label{question-add-details-to-the-derivations-of-kf36}

\begin{itemize}
\item
  The prediction recursions {[}KF3--4{]} are relatively easy to
  demonstrate, but it is a good exercise to go through the algebra to
  your own satisfaction.
\item
  A useful trick for the algebra is to notice that the conditioning
  identities {[}KF1{]} for joint Gaussian random variables continue to
  hold if left and right are both conditioned on some additional jointly
  Gaussian variable, such as \(Y_{1:n-1}\).
\item
  {[}KF5--6{]} can be deduced by completing the square in an expression
  for the joint density,
  \(f_{X_nY_n|Y_{1:n-1}}(x_n,y_n\given y_{1:n-1})\) and noticing that
  the marginal density of \(X_n\) given \(Y_{1:n}\) is proportional to
  the joint density, with a normalizing constant that must allow the
  marginal density to integrate to one.
\end{itemize}

\begin{center}\rule{0.5\linewidth}{\linethickness}\end{center}

\begin{center}\rule{0.5\linewidth}{\linethickness}\end{center}


\end{document}
